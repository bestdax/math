\section{SVSU2022 level1}
\begin{questions}
	\question 考虑方程$p(x): ax^2 + bx + c=0$,其系数$a$,$b$和$c$都是非零的,并且每个系数都满足从方程$p(x)$
	中移除包含该系数的项后得到的方程;例如,系数$b$是方程 $ax^2+c=0$的一个解。求方程$p(x)$
	所有解的和是多少?

	\begin{oneparchoices}
		\CorrectChoice 总是 1 \choice 总是 -1 \choice 总是 2 \choice 1 或者 -1 \choice 1 或者 2
	\end{oneparchoices}
	\begin{solution}
		方程$p(x)$的两个解分别为$\displaystyle x_1, x_2 = \frac{-b \pm \sqrt{b^2 - 4ac}}{2a}$,则$\displaystyle x_1 +
			x_2 = -\frac{b}{a}$\\
		由题意得
		\begin{align}
			ac^2  + bc & = 0 \label{\thequestion:1} \\
			ab^2  + c  & = 0 \label{\thequestion:2} \\
			ab    + c  & = 0 \label{\thequestion:3}
		\end{align}
		由式(\ref{\thequestion:2}) 和(\ref{\thequestion:3})得$b=1$和$a+c=0$,代入(\ref{\thequestion:1})得:
		\begin{equation}
			a(-a) -a  = 0
		\end{equation}

		则$a=-1$或$a=0$,因为题目中$a \ne 0$,所以$a=-1$,则$\displaystyle x_1 + x_2 = -\frac{b}{a}=1$
	\end{solution}
	\question 一支行进乐队共有150名成员。有一天,只有部分成员到场,但没有人想花时间去清点具体人数。他们首先尝试以每排5名成员的方式排队,但结果多出了1名成员。接着他们改以每排6名成员的方式排列,仍然多出了1名成员。最后,他们尝试以每排7名成员的方式排列,这次多出了2名成员。请问,为了确保没有成员被剩下,他们应该以每排多少人的方式排列?

	\begin{oneparchoices}
		\choice 4人每排
		\CorrectChoice 11人每排
		\choice 13人每排
		\choice 17人每排
		\choice 上述选项中无正确答案
	\end{oneparchoices}
	\begin{solution}
		设当天一共到场$x$人, 根据题意得
		\begin{equation}
			\begin{cases}
				x \equiv 1 \pmod 5 \\
				x \equiv 1 \pmod 6
			\end{cases}
		\end{equation}
		则$x=k(5 \times 6) + 1$,其中$k=0,1,2,3 \ldots$

		根据总人数小于150这个条件,$x$可能的值有$1,31,61,91,121$,然后根据每排排7人的情况下多出2名成员,得到$x=121$,则每排11个人可以确保没有成员被剩下。

	\end{solution}

	\question 下列陈述中,哪一个是唯一正确的?
	\begin{choices}
		\choice 水平线的图形不可能有任何与x轴的交点。
		\CorrectChoice 水平线的图形不能有一个唯一的与x轴的交点,但可能有超过一个与x轴的交点。
		\choice 抛物线的图形不能有一个唯一的与x轴的交点,但可能有超过一个与x轴的交点。
		\choice 三次或更高次的多项式的图形必须至少有一个与x轴的交点。
		\choice 要么没有一个是正确的,要么不止一个正确。
	\end{choices}
	\begin{solution}\\
		选项A:当$x=0$时与$x$轴有无穷多个交点。\\
		选项B:根据以上,此描述正确。\\
		选项C:当抛物线的顶点位于$x$轴上时有唯一的交点,所以此描述错误。\\
		选项D:比如$x^4+1$这样的曲线就没有与$x$轴的交点,所以此描述错误。
	\end{solution}

	\question 下列哪个选项等于10的阶乘($10!$)。如果你对符号不熟悉,$n!$(读作\enquote{n的阶乘})对于任何非负整数$n$定义如下:
	\begin{equation*}
		\begin{cases}
			0! = 1 \\
			n > 0,n! = n \cdot (n-1)!
		\end{cases}
	\end{equation*}
	你需要从下面的选项中选择:

	\begin{oneparchoices}
		\choice $5! \cdot 2!$
		\CorrectChoice $7! \cdot 5! \cdot 3!$
		\choice $7! \cdot 5! \cdot 2!$
		\choice $7! \cdot 5! \cdot 3! \cdot 2!$ \\
		\choice 上述选项中没有符合条件的,或者有超过一个选项符合条件。
	\end{oneparchoices}

	\begin{solution}
		\begin{align*}
			10! & = 7! \cdot 8 \cdot 9 \cdot 10                        \\
			    & = 7! \cdot 2 \cdot 4 \cdot 3 \cdot 3 \cdot 2 \cdot 5 \\
			    & = 7! \cdot 5! \cdot 3!
		\end{align*}
	\end{solution}

	\question 一个数的六进制表示为550,另一个数的五进制表示为3440,求它们的最大公约数用4进制表示为多少。

	\begin{oneparchoices}
		\choice 24 \CorrectChoice 33 \choice 113 \choice 223 \choice 以上都不对
	\end{oneparchoices}

	\begin{solution}
		\begin{align*}
			550_{(6)}                    & = 5 \times 6^2 + 5 \times 6^1 + 0 \times 6^0                \\
			                             & = 210_{(10)}                                                \\
			3440_{(5)}                   & = 3 \times 5^3 + 4 \times 5^2 + 4 \times 5^1 + 0 \times 5^0 \\
			                             & = 495_{(10)}                                                \\
			\gcd(210_{(10)}, 495_{(10)}) & = 15_{(10)}                                                 \\
			                             & = 3 \times 4^1 + 3 \times 4^0                               \\
										 & = 33_{(4)}
		\end{align*}

	\end{solution}

\end{questions}
\section{SVSU2022 level2}
\subsection{}
求
$\displaystyle\sum_{k=0}^{2022}\frac{2022!\cdot (-1)^k2^k}{k! \cdot (2022-k)!}$
的值

\subsection*{解:}
\begin{enumerate}
	\item 利用二项式定理$\displaystyle(x+y)^n = \sum_{k=0}^{n}\binom{n}{k}x^ky^{n-k}$
	      \begin{align}
		      \sum_{k=0}^{2022}\frac{2022!\cdot(-1)^k2^k}{k!\cdot(2022-k)!} & = \sum_{k=0}^{2022}\frac{2022!}{k!\cdot
		      (2022-k)!}(-2)^k                                                                                        \\
		                                                                    & =
		      \sum_{k=0}^{2022}\binom{2022}{k}(-2)^k\cdot 1^{2022-k}                                                  \\
		                                                                    & = (1-2)^{2022}                          \\
		                                                                    & = 1
	      \end{align}
\end{enumerate}

\subsection{}
求$(\sqrt[4]{27} + \sqrt{3} + \sqrt[4]{3} + 1)^2$
\subsection*{解:}
\begin{enumerate}
	\item 设$\displaystyle x=3^{\frac{1}{4}}$,则
	      \begin{align}
		      (\sqrt[4]{27} + \sqrt{3} + \sqrt[4]{3} + 1)^2 & = (x^3 + x^2 + x + 1)^2                     \\
		                                                    & = [x^2(x+1) + (x+1)]^2                      \\
		                                                    & = (x^2 + 1)^2(x+1)^2                        \\
		                                                    & = \frac{(x^2 + 1)^2(x+1)^2(x-1)^2}{(x-1)^2} \\
		                                                    & = \frac{(x^4 - 1)^2}{(x-1)^2}
	      \end{align}
	\item 将$x=\sqrt[4]{3}$代入得:
	      \begin{align}
		      \frac{(x^4 - 1)^2}{(x-1)^2} & = \frac{4}{(x-1)^2}                     \\
		                                  & = \frac{4}{x^2 - 2x + 1}                \\
		                                  & = \frac{4}{\sqrt{3} - 2\sqrt[4]{3} + 1}
	      \end{align}
\end{enumerate}
