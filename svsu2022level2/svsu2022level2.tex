%!tex program = lualatex
\documentclass[answers]{exam}
\usepackage{ctex}
\usepackage{graphicx}
\usepackage[margin=2cm]{geometry}
\usepackage{amsmath, amssymb}
\usepackage{csquotes}
\usepackage{tikz, pgfplots}
\usetikzlibrary{
	angles,
	backgrounds,
	calc,
	decorations.pathmorphing,
	decorations.pathreplacing,
	decorations.text,
	intersections,
	patterns,
	quotes,
	shapes,
	shapes.symbols,
}
\pagestyle{empty}
\newcounter{xcord}
\newcounter{ycord}
\newcounter{total}
\renewcommand{\labelenumi}{\textbf{\ifnum\value{enumi}<10 0\fi\arabic{enumi})}}

\pgfplotsset{compat=1.18}

\CorrectChoiceEmphasis{\color{blue!70!green}\bfseries}
\renewcommand{\solutiontitle}{\textbf{解:}}

\usepackage{array, tabularx}
\newcolumntype{C}{>{\centering\arraybackslash}X}
\newcolumntype{B}{>{\centering\bfseries\arraybackslash}X}
\catcode`\幺=0

\begin{document}
\begin{questions}
	\question 对于 \( n \ge 2 \),让 \( k_n = 5 \cdot 10 \cdot 17 \cdot 26 \cdot\cdots\cdot (n^2 + 1) \),
	\begin{equation*}
		S_n = (1-\frac{1}{2^4})(1-\frac{1}{3^4})\cdots(1-\frac{1}{n^4}).
	\end{equation*}
	那么 \( S_n / k_n \)等于:

	\begin{oneparchoices}
		\CorrectChoice \(\displaystyle \frac{1}{2}\frac{1}{(n!)^2}(1+\frac{1}{n}) \)
		\choice \(\displaystyle \frac{1}{k_n}\frac{n^2}{n^4 + 1} \)
		\choice \(\displaystyle \frac{n^4 + 1}{2n+2}\cdot \frac{1}{n!} \)
		\choice \(\displaystyle \frac{1}{k_n}\frac{1}{n^4}\frac{1}{((n-1)!)^4} \)
		\choice 以上都不对
	\end{oneparchoices}

	\begin{solution}
		\begin{align*}
			S_n / k_n & = \frac{(1-\frac{1}{2^4})(1-\frac{1}{3^4})\cdots(1-\frac{1}{n^4})}{5 \cdot 10 \cdot 17 \cdot 26
			\cdot\cdots\cdot (n^2 + 1)}                                                                                 \\
			          & = \prod_2^n\frac{1 - \frac{1}{n^4}}{n^2 + 1}                                                    \\
			          & = \prod_2^n\frac{n^4 - 1}{n^4(n^2+1)}                                                           \\
			          & = \prod_2^n\frac{n^2 - 1}{n^4}                                                                  \\
			          & = \prod_2^n\frac{1 - \frac{1}{n^2}}{n^2}                                                        \\
			          & = \frac{1}{(n!)^2} \prod_2^n(1-\frac{1}{n^2})                                                   \\
			          & = \frac{1}{(n!)^2} \prod_2^n\frac{(n-1)(n+1)}{n^2}                                              \\
			          & = \frac{1}{(n!)^2} \frac{(2-1)(2+1)(3-1)(3+1)\cdots
			(n-3)(n-1)(n-2)n(n-1)(n+1)}{2^2\cdot3^2\cdots (n-1)^2n^2}                                                   \\
			          & = \frac{1}{(n!)^2} \frac{1\cdot2\cdot 3^2 \cdot 4^2 \cdots (n-1)^2 \cdot n \cdot
			(n+1)}{2^2\cdot3^2\cdots (n-1)^2\cdot n^2}                                                                  \\
			          & = \frac{1}{(n!)^2} \frac{2n(n+1)}{2^2\cdot n^2}                                                 \\
			          & = \frac12\frac1{(n!)^2}(1+\frac1n)
		\end{align*}
	\end{solution}

	\question 下列选项中,哪个不是方程:
	\begin{equation*}
		3^x - 6^x = 12^x - \left(\frac32\right)^x.
	\end{equation*}
	的解?

	\begin{oneparchoices}
		\choice \( \dfrac{3\pi i}{\ln2} \)
		\choice 0
		\choice \( \dfrac{6\pi i}{\ln2} \)
		\choice \( \dfrac{27\pi i}{\ln2} \)
		\CorrectChoice 所有都是解
	\end{oneparchoices}

	\begin{solution}
		\begin{align*}
			3^x - 6^x                               & = 12^x - \left(\frac32\right)^x             \\
			3^x - 2^x \cdot 3^x                     & = 2^x \cdot 2^x \cdot 3^x - \frac{3^x}{2^x} \\
			2^x \cdot 3^x - 2^x \cdot 2^x \cdot 3^x & = 2^x \cdot 2^x \cdot 2^x \cdot 3^x - 3^x
		\end{align*}
		设 \( a = 2^x, b = 3^x \),则有
		\begin{align*}
			a^3b + a^2b - ab - b & = 0
		\end{align*}
		因为 \( b \ne 0 \),所以进一步化简得
		\begin{align*}
			a^3 + a^2 - a - 1 & = 0 \\
			a^2(a+1) - (a+1)  & = 0 \\
			(a+1)(a^2 - 1)    & = 0 \\
			(a+1)^2(a-1)      & = 0
		\end{align*}
		则有 \( a = \pm 1 \)即 \( 2^x = \pm 1 \)

		欧拉公式 \( e^{i\theta} = \cos\theta + i\sin\theta \)对于 \( \theta = k\pi, k \in \mathbb{Z}  \)时有
		\begin{equation*}
			\begin{cases}
				\cos(k\pi) + i\sin(k\pi) = 1, k \text{是偶数} \\
				\cos(k\pi) + i\sin(k\pi) = -1, k \text{是奇数}
			\end{cases}
		\end{equation*}
		将 \( 2^x = e^{x\ln2} \)代入欧拉公式有
		\begin{gather*}
			x \ln 2 = k\pi i,  k \in \mathbb{Z} \\
			x = \frac{k\pi i}{\ln2}, k \in \mathbb{Z}
		\end{gather*}
		所以选E
	\end{solution}

	\question 考虑方程$p(x): ax^2 + bx + c=0$,其系数$a$,$b$和$c$都是非零的,并且每个系数都满足从方程$p(x)$
	中移除包含该系数的项后得到的方程;例如,系数$b$是方程 $ax^2+c=0$的一个解。求方程$p(x)$
	所有解的和是多少?

	\begin{oneparchoices}
		\choice 总是 1 \choice 总是 -1 \choice 总是 2 \CorrectChoice 1 或者 -1 \choice 1 或者 2
	\end{oneparchoices}
	\begin{solution}
		方程$p(x)$的两个解分别为$\displaystyle x_1, x_2 = \frac{-b \pm \sqrt{b^2 - 4ac}}{2a}$,则$\displaystyle x_1 +
			x_2 = -\frac{b}{a}$\\
		由题意得
		\begin{align}
			ac^2  + bc & = 0 \label{\thequestion:1} \\
			ab^2  + c  & = 0 \label{\thequestion:2} \\
			ab    + c  & = 0 \label{\thequestion:3}
		\end{align}
		由式(\ref{\thequestion:2}) 和(\ref{\thequestion:3})得$b=1$和$a+c=0$,代入(\ref{\thequestion:1})得:
		\begin{equation}
			a(-a)^2 -a  = 0
		\end{equation}

		则$a=\pm 1$或$a=0$,因为题目中$a \ne 0$,所以$a=\pm 1$,则$\displaystyle x_1 + x_2 = -\frac{b}{a}=\pm 1$

		此题与2022level1中重复
	\end{solution}

	\question 求下列序列前五项的和:
	\begin{equation*}
		\left\{ \dfrac{2}{3}, \dfrac{4}{9}, \dfrac{8}{27}, \dots\right\}
	\end{equation*}

	\begin{oneparchoices}
		\choice \( \dfrac{64}{81} \)
		\choice \( \dfrac{550}{243} \)
		\choice \( \dfrac{32}{81} \)
		\CorrectChoice \( \dfrac{422}{243} \)
		\choice 以上都不对
	\end{oneparchoices}

	\begin{solution}
		设 \( S_5 = \dfrac{2}{3} + \left(\dfrac{2}{3}\right)^2 + \dots + \left(\dfrac{2}{3}\right)^5\)
		则有
		\begin{align*}
			\dfrac{2}{3}\cdot S_5 & = \left(\dfrac{2}{3}\right)^2 + \dots + \left(\dfrac{2}{3}\right)^6 \\
			\dfrac{1}{3}\cdot S_5 & = \dfrac{2}{3} - \left(\dfrac{2}{3}\right)^6                        \\
			S_5                   & = 2 - \dfrac{2^6}{3^5}                                              \\
			                      & = \dfrac{243 \times 2 - 64}{243}                                    \\
			                      & = \dfrac{422}{243}
		\end{align*}
	\end{solution}

	\question 假设 \( f \)是一个函数,并有 \( f(n+1) = f(n) + 3f(n-1) \)对所有整数成立。如果 \( f(4) = 151 \)和 \( f(1)
	= 1\),那么 \( f(-1) \)的值是多少?

	\begin{oneparchoices}
		\choice 12 \choice 4 \CorrectChoice \( -\dfrac{11}{3} \) \choice \( \dfrac{13}{3} \) \choice 以上都不对
	\end{oneparchoices}

	\begin{solution}
		由题意得
		\begin{align*}
			f(4)  & = f(3) + 3f(2)           \\
			f(3)  & = f(2) + 3f(1)           \\
			f(4)  & = f(2) + 3f(1) + 3f(2)   \\
			151   & = 4f(2) + 3              \\
			f(2)  & = \dfrac{151 -3}{4} = 37 \\
			      & = f(1) + 3f(0)           \\
			f(0)  & = 12                     \\
			f(1)  & = f(0) + 3f(-1)          \\
			f(-1) & = -\dfrac{11}{3}
		\end{align*}
	\end{solution}

	\question 对下式进行分母有理化
	\begin{equation*}
		\dfrac{1}{2^{5/6} + 2^{1/2} + 2^{1/3} + 1}
	\end{equation*}

	\begin{oneparchoices}
		\choice \( \dfrac{1}{6}(2^{1/2} + 2^{1/3} - 2^{1/6}) \)
		\choice \( \dfrac{1}{3}(2^{1/2} - 1)(2^{2/3} - 2^{1/3} + 1) \)
		\choice \( \dfrac{1}{2}(2^{7/6} - 2^{5/6} + 1) \)
		\CorrectChoice \( (2^{1/2} - 1)(2^{2/3} - 2^{1/3} + 1) \)
		\choice 以上都不对
	\end{oneparchoices}

	\begin{solution}
		\begin{align*}
			\dfrac{1}{2^{5/6} + 2^{1/2} + 2^{1/3} + 1} & = \dfrac{1}{(2^{1/6})^5 + (2^{1/6})^3 + (2^{1/6})^2 + 1},
			\qquad \text{设} a = 2^{1/6}                                                                              \\
			                                           & = \dfrac{1}{a^5 + a^3 + a^2 + 1}                             \\
			                                           & = \dfrac{1}{a^3(a^2 + 1) + a^2 + 1}                          \\
			                                           & = \dfrac{1}{(a^3 + 1)(a^2 + 1)}                              \\
			                                           & = \dfrac{a^3 - 1}{(a^6 - 1)(a^2 + 1)}                        \\
			                                           & = \dfrac{a^3 - 1}{a^2 + 1}                                   \\
			                                           & = \dfrac{(a^3 - 1)(a^4 - a^2 + 1)}{(a^2 + 1)(a^4 - a^2 + 1)} \\
			                                           & = (a^3 - 1)(a^4 - a^2 + 1)                                   \\
			                                           & = (2^{1/2} - 1)(2^{2/3} - 2^{1/3} + 1)
		\end{align*}
	\end{solution}

	\question 求\(  \displaystyle\sum_{k=0}^{2022}\frac{2022!\cdot (-1)^k2^k}{k! \cdot (2022-k)!}\) 的值

	\begin{oneparchoices}
		\choice \( \dfrac{1-(-2)^{2023}}{3} \)
		\choice -1
		\choice 1
		\choice \( 3^{2022} \)
		\choice 以上都不对
	\end{oneparchoices}

	\begin{solution}
		\begin{enumerate}
			\item 利用二项式定理$\displaystyle(x+y)^n = \sum_{k=0}^{n}\binom{n}{k}x^ky^{n-k}$
			      \begin{align*}
				      \sum_{k=0}^{2022}\frac{2022!\cdot(-1)^k2^k}{k!\cdot(2022-k)!} & = \sum_{k=0}^{2022}\frac{2022!}{k!\cdot
				      (2022-k)!}(-2)^k                                                                                        \\
				                                                                    & =
				      \sum_{k=0}^{2022}\binom{2022}{k}(-2)^k\cdot 1^{2022-k}                                                  \\
				                                                                    & = (1-2)^{2022}                          \\
				                                                                    & = 1
			      \end{align*}
		\end{enumerate}
	\end{solution}

	\question $(\sqrt[4]{27} + \sqrt{3} + \sqrt[4]{3} + 1)^2$ 与下列哪个相等

	\begin{oneparchoices}
		\choice \( \dfrac{4}{\sqrt{3} - 2\sqrt[4]{3} + 1} \)
		\choice \( 4\sqrt{3} + 4 \)
		\choice \( 2\sqrt[4]{27} + 2\sqrt[4]{3} + 6\sqrt{3} + 6 \)
		\choice \( \dfrac{16}{\sqrt{3} + 2\sqrt[4]{3} + 1} \)
		\choice 以上都不对
	\end{oneparchoices}

	\begin{solution}
		\begin{enumerate}
			\item 设$\displaystyle x=3^{\frac{1}{4}}$,则
			      \begin{align*}
				      (\sqrt[4]{27} + \sqrt{3} + \sqrt[4]{3} + 1)^2 & = (x^3 + x^2 + x + 1)^2                     \\
				                                                    & = [x^2(x+1) + (x+1)]^2                      \\
				                                                    & = (x^2 + 1)^2(x+1)^2                        \\
				                                                    & = \frac{(x^2 + 1)^2(x+1)^2(x-1)^2}{(x-1)^2} \\
				                                                    & = \frac{(x^4 - 1)^2}{(x-1)^2}
			      \end{align*}
			\item 将$x=\sqrt[4]{3}$代入得:
			      \begin{align*}
				      \frac{(x^4 - 1)^2}{(x-1)^2} & = \frac{4}{(x-1)^2}                     \\
				                                  & = \frac{4}{x^2 - 2x + 1}                \\
				                                  & = \frac{4}{\sqrt{3} - 2\sqrt[4]{3} + 1}
			      \end{align*}
		\end{enumerate}
	\end{solution}
\end{questions}
\end{document}
