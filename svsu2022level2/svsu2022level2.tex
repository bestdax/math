%!tex program = lualatex
\documentclass[answers]{exam}
\usepackage{ctex}
\usepackage{graphicx}
\usepackage[margin=2cm]{geometry}
\usepackage{amsmath, amssymb}
\usepackage{csquotes}
\usepackage{tikz, pgfplots}
\usetikzlibrary{
	angles,
	backgrounds,
	calc,
	decorations.pathmorphing,
	decorations.pathreplacing,
	decorations.text,
	intersections,
	patterns,
	quotes,
	shapes,
	shapes.symbols,
}
\pagestyle{empty}
\newcounter{xcord}
\newcounter{ycord}
\newcounter{total}
\renewcommand{\labelenumi}{\textbf{\ifnum\value{enumi}<10 0\fi\arabic{enumi})}}

\pgfplotsset{compat=1.18}

\CorrectChoiceEmphasis{\color{blue!70!green}\bfseries}
\renewcommand{\solutiontitle}{\textbf{解:}}

\usepackage{array, tabularx}
\newcolumntype{C}{>{\centering\arraybackslash}X}
\newcolumntype{B}{>{\centering\bfseries\arraybackslash}X}
\catcode`\幺=0

\begin{document}
\begin{questions}
	\question 对于 \( n \ge 2 \),让 \( k_n = 5 \cdot 10 \cdot 17 \cdot 26 \cdot\cdots\cdot (n^2 + 1) \),
	\begin{equation*}
		S_n = (1-\frac{1}{2^4})(1-\frac{1}{3^4})\cdots(1-\frac{1}{n^4}).
	\end{equation*}
	那么 \( S_n / k_n \)等于:

	\begin{oneparchoices}
		\CorrectChoice \(\displaystyle \frac{1}{2}\frac{1}{(n!)^2}(1+\frac{1}{n}) \)
		\choice \(\displaystyle \frac{1}{k_n}\frac{n^2}{n^4 + 1} \)
		\choice \(\displaystyle \frac{n^4 + 1}{2n+2}\cdot \frac{1}{n!} \)
		\choice \(\displaystyle \frac{1}{k_n}\frac{1}{n^4}\frac{1}{((n-1)!)^4} \)
		\choice 以上都不对
	\end{oneparchoices}

	\begin{solution}
		\begin{align*}
			S_n / k_n & = \frac{(1-\frac{1}{2^4})(1-\frac{1}{3^4})\cdots(1-\frac{1}{n^4})}{5 \cdot 10 \cdot 17 \cdot 26
			\cdot\cdots\cdot (n^2 + 1)}                                                                                 \\
			          & = \prod_2^n\frac{1 - \frac{1}{n^4}}{n^2 + 1}                                                    \\
			          & = \prod_2^n\frac{n^4 - 1}{n^4(n^2+1)}                                                           \\
			          & = \prod_2^n\frac{n^2 - 1}{n^4}                                                                  \\
			          & = \prod_2^n\frac{1 - \frac{1}{n^2}}{n^2}                                                        \\
			          & = \frac{1}{(n!)^2} \prod_2^n(1-\frac{1}{n^2})                                                   \\
			          & = \frac{1}{(n!)^2} \prod_2^n\frac{(n-1)(n+1)}{n^2}                                              \\
			          & = \frac{1}{(n!)^2} \frac{(2-1)(2+1)(3-1)(3+1)\cdots
			(n-3)(n-1)(n-2)n(n-1)(n+1)}{2^2\cdot3^2\cdots (n-1)^2n^2}                                                   \\
			          & = \frac{1}{(n!)^2} \frac{1\cdot2\cdot 3^2 \cdot 4^2 \cdots (n-1)^2 \cdot n \cdot
			(n+1)}{2^2\cdot3^2\cdots (n-1)^2\cdot n^2}                                                                  \\
			          & = \frac{1}{(n!)^2} \frac{2n(n+1)}{2^2\cdot n^2}                                                 \\
			          & = \frac12\frac1{(n!)^2}(1+\frac1n)
		\end{align*}
	\end{solution}

	\question 求\(  \displaystyle\sum_{k=0}^{2022}\frac{2022!\cdot (-1)^k2^k}{k! \cdot (2022-k)!}\) 的值
	\begin{solution}
		\begin{enumerate}
			\item 利用二项式定理$\displaystyle(x+y)^n = \sum_{k=0}^{n}\binom{n}{k}x^ky^{n-k}$
			      \begin{align}
				      \sum_{k=0}^{2022}\frac{2022!\cdot(-1)^k2^k}{k!\cdot(2022-k)!} & = \sum_{k=0}^{2022}\frac{2022!}{k!\cdot
				      (2022-k)!}(-2)^k                                                                                        \\
				                                                                    & =
				      \sum_{k=0}^{2022}\binom{2022}{k}(-2)^k\cdot 1^{2022-k}                                                  \\
				                                                                    & = (1-2)^{2022}                          \\
				                                                                    & = 1
			      \end{align}
		\end{enumerate}
	\end{solution}

	\question 求$(\sqrt[4]{27} + \sqrt{3} + \sqrt[4]{3} + 1)^2$

	\begin{solution}
		\begin{enumerate}
			\item 设$\displaystyle x=3^{\frac{1}{4}}$,则
			      \begin{align}
				      (\sqrt[4]{27} + \sqrt{3} + \sqrt[4]{3} + 1)^2 & = (x^3 + x^2 + x + 1)^2                     \\
				                                                    & = [x^2(x+1) + (x+1)]^2                      \\
				                                                    & = (x^2 + 1)^2(x+1)^2                        \\
				                                                    & = \frac{(x^2 + 1)^2(x+1)^2(x-1)^2}{(x-1)^2} \\
				                                                    & = \frac{(x^4 - 1)^2}{(x-1)^2}
			      \end{align}
			\item 将$x=\sqrt[4]{3}$代入得:
			      \begin{align}
				      \frac{(x^4 - 1)^2}{(x-1)^2} & = \frac{4}{(x-1)^2}                     \\
				                                  & = \frac{4}{x^2 - 2x + 1}                \\
				                                  & = \frac{4}{\sqrt{3} - 2\sqrt[4]{3} + 1}
			      \end{align}
		\end{enumerate}
	\end{solution}
\end{questions}
\end{document}
